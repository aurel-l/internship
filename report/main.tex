\documentclass[a4paper,11pt]{report}

%packages
\usepackage[utf8]{inputenc}

%document
\begin{document}
\begin{titlepage} 
  \parindent0pt 
  \hskip.25cm \vrule width.8pt \hskip.25cm 
  \begin{minipage}{\dimexpr\linewidth*2-2.5cm-.8pt-2.5cm\relax} 
    \sffamily\bfseries {
    \huge Master 2 Bioinformatique\par}\large 
    Semestre 10\par Université de Bordeaux\par 
    %%Année truc
  \end{minipage}%
  \vskip10pt plus 1fil
  \fbox{%
  \begin{minipage}{\dimexpr\linewidth-2\fboxsep-2\fboxrule} 
    \centering\Large\bfseries 
    \vskip1cm 
    Project report: TITLE\par 
    \vskip1cm \kern0pt 
  \end{minipage}%
  }%
  \vskip0.5cm 
  {\centering 
  \vskip0pt plus 1fil 
  }%

  \textbf{Aurélien \sc{Luciani}}\centering\\ 
  \vskip0pt plus 1fil 
  \hfill \hskip.25cm \vrule width.8pt \hskip.25cm 
  \vskip0pt plus 1fil {\centering Supervisor\,: Murray P. \sc{Cox} \par }
  \vskip0pt plus 2fil 
  \hfill\bfseries{Year 2014-2015}\hfill\null 
\end{titlepage}

\tableofcontents

\chapter*{Introduction}
\addcontentsline{toc}{chapter}{Introduction}
Two groups of populations can be identified in the Islands of South-East Asia (ISEA), one is composed of the Melanesians, whose ancestors settled in these islands during the first human settlement, around 45 thousand years ago. The other arrived more recently during a period often called the Austronesian expansion, between 5 and 4 thousand years ago, when people from mainland China settled in the islands. Nowadays people living in this area have mixed genomic ancestry and markers can be identified and defined as either from an Asian ancestry or a Melanesian one. These markers are based on signle nucleotid polymorphisms located in different chromosomes and 52 markers can be used to define accurately the admixture of the Asian ancestry in every individuals.

The choice of these SNPs is a result of previous studies at Massey University, the University of Arizona, the Santa Fe Institute, and the Eijkman Institute that sequenced 1430 individuals from 60 populations. This set of SNPs is defined as highly informative and allows to define the ancestry of a person based on a small quantity of markers that are highly discriminant.

Two specific patterns can be seen, one is the non linear gradient of Asian admixture when observing individuals in the different islands when looking along the longitudinal axis, corresponding more or less the wave the settling might have happened. The second is the difference of admixture when looking at specific parts of the genomes associated with male or female ancestry, implying a gender-biased expansion.
% first papers
% whole project for the group
%% leading to the creation of the model


\chapter{Project presentation}
The project consists of developing a model of the Austronesian expansion throughout the ISEA that could reproduce the same two patterns observed in the real data.
The first pattern can be seen in the figure (REF!!!) and 
% model details
% previous works
%% previous models
%% previous analyses


\chapter{Implementation}
% or "realisation", "analysis pipeline"

% run management (local, cluster, NeSI)
% data processing from model output
% data storage, adding and querying data
% analysis
%% comparisons
%% visualisations
% ABC framework


\chapter{Results}
\section{Grid search analysis}


\section{ABC Framework}


\chapter{Discussion}
% optimisations
% other visualisations
% more complex model


\chapter*{Conclusion}
\addcontentsline{toc}{chapter}{Conclusion}

\begin{thebibliography}{99}
\addcontentsline{toc}{chapter}{Bibliographie}


\end{thebibliography}

\end{document}


